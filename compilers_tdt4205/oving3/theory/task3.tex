\subsection{Task 1.3: Syntax-Directed Translation}


\subsubsection{}
A syntax directed definition is a parser where the productions can have semantic rules, and nonterminals with attributes.
This is useful when we want to use the parser for semantic analysis and translation. 

\subsubsection{}
S-attributed syntax directed definitions only use synthesized attributes, which means all attributes on the left side of a production may only use attributes from the right-hand side (attributes may only be passed up the parse tree).
L-attributed SDDs may also use attributes of parent nodes when evaluating atttributes, e.g. a right-hand side attribute may use attributes from the left-hand side.
GNU Bison does not support L-attributed SDDs because it generates bottom-up parser, which means the children nodes are generated before the parent nodes, and therefore can not depend on attributes stored in parent nodes.

\subsubsection{}

\begin{description}
 \item[] $E \rightarrow E + T | T$
 \item[] $T \rightarrow num , num | num$
\end{description}

\begin{lstlisting}
E:  E1 + T      { if(E1.type == FLOAT || T.type == FLOAT) then;
			E.type = FLOAT
		  else
	                E.type = INTEGER
		  fi}
  | T		{ E.type = T.type }
  ;
 
T:  num "," num { T.type = FLOAT, T.value = float(num) }
  | num         { T.type = INTEGER, T.value = int(num) }
  ;

\end{lstlisting}






