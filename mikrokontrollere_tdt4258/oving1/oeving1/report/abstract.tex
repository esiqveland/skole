The STK1000 developer board and AVR-32 based microprocessor (both made bu Atmel)  will be used for the remaining exercises, and getting to know how to develop for these systems is important. Developing for micro controllers differs from other development in that code needs to be compiled for an other system and that debugging needs to happen remotely. There exists tools to make this easy.
Assembly language is normally the language closest to machine code that programmers have to work with. It consist mostly of instructions that the processors can execute directly (once translated). Understanding it is important in understanding hoe processors implement higher functions.
In this paper we will describe a program written in assembly for the STK1000 with an AVR-32 based microcontroller. This program initialises a row of 8 led lights as output and a row of 8 buttons as input. It then turns on one of the lights and enables interrupts on two of the buttons. It then goes to sleep. Using an interrupt routine, it moves the led that is currently on to the left or right when the left or right button is pressed. Additionally, upon reaching the end, it will circle around. The interrupt routine implements an anti-bouncing loop.
